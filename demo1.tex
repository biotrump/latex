\documentclass[11pt]{article}
\usepackage{mathtools}
\usepackage{letltxmacro} 
%Gummi|061|=)
\title{\textbf{Welcome to Gummi 0.6.1}}
\author{Alexander van der Mey\\
		Wei-Ning Huang\\
		Dion Timmermann}
\date{}
\begin{document}

\maketitle

\section{Before you start}

inline:$ \frac{2}{\|w\|}$ \\[2em]

Display:
\[ \frac{2}{\|w\|} \]

Numbered Display:
\begin{equation}
 \frac{2}{\|w\|} 
Cov[AX] = ACov[X]A^{T}\\\\
Cov[X] = \sum = U \Lambda U^{T},\ \Lambda=diag(\lambda i i)=\Lambda ^T\\\\
We\ define:\\
Y = \Lambda^{-1/2} U^{T} X,\\
let\ W = \Lambda^{-1/2} U^{T}.\ We\ call\ W\ "whitening"\ matrix.\\
Y = WX\\
\Lambda^{-1/2} = diag(1/\sqrt{\Lambda i i})\\\\
Cov[Y] = Cov[(\Lambda^{-1/2} U^{T}) X]=
(\Lambda^{-1/2} U^{T}) Cov[X] (\Lambda^{-1/2} U^{T}) ^T=\\\\
(\Lambda^{-1/2} U^{T})  U \Lambda U^{T} (\Lambda^{-1/2} U^{T}) ^T=
\Lambda^{-1/2} (U^{T}  U) \Lambda U^{T} (\Lambda^{-1/2} U^{T}) ^T=\\
\Lambda^{-1/2} (I) \Lambda U^{T} ({U^{T}}^T{\Lambda^{-1/2}}^T)=
\Lambda^{-1/2} \Lambda U^{T} (U{\Lambda^{-1/2}}^T)=
\Lambda^{-1/2} \Lambda (U^{T} U) {\Lambda^{-1/2}}^T=\\
\Lambda^{-1/2} \Lambda (I) {\Lambda^{-1/2}}^T=
\Lambda^{-1/2} \Lambda \Lambda^{-1/2}=\\
\end{equation}

\begin{displaymath}
  MR(e) = \prod_{(R_i\cdot ID, R_j\cdot ID)} \\
  \sqrt{\frac{a}{b}} \\
  Y = \Lambda^{-1/2} U^{T} X,
\end{displaymath}

\section{Weighted Mean Squared Error: WMSE}
A training data is a set $\{(I_s,v_s,w_s) : s=1,2,..,S\}$
where $v_s$ is the \textbf{ground truth value} associated with image $I_s$ and $w_s$ is its weight.
\\ A pixel intensity comparison binary test on a image I is defined as

\begin{equation}
\label{eq:bintest}
    bintest(I;l_1,l_2)= 
\begin{cases}
    0,		& \text{if } I(l_1)\leq I(l_2)\\
    1,      & \text{otherwise}
\end{cases}
\end{equation}\\
\\ A \textbf{bintest}, $f_i$, has \textbf{only two values: 0 or 1}, so a image will be classified as 0 or 1 by the \textbf{bintest}. If an image is classified to "0", it's in class/cluster $C_0$, otherwise it's in class/cluster $C_1$. \\
\\ Scalar $v_0$ and $v_1$ are \textbf{weighted averages of the ground truths} in $C_0$ and $C_1$, respectively.

\begin{equation}
\label{eq:WAGT0}
\bar{v_0}=\frac{\sum\nolimits_{(I,v,w) \in C_0} w_i v_i} { \sum\nolimits_{(I,v,w) \in C_0} w_i} =
\end{equation}
\[
\frac{\sum_{i=1}^{\| C_0 \|}  w_i v_i}
{\sum_{i=1}^{\| C_0 \|}  w_i}
\]
\begin{equation}
\label{eq:WAGT1}
\bar{v_1}=\frac{\sum\nolimits_{(I,v,w) \in C_1} w_i v_i} { \sum\nolimits_{(I,v,w) \in C_1} w_i}=
\end{equation}
\[
\frac{\sum_{i=1}^{\| C_1 \|}  w_i v_i}
{\sum_{i=1}^{\| C_1 \|}  w_i}
\]

The ground truth, $v_s$, of a positive sample is annotated with $+1$ and 
a negative sample is annotated with $-1$.\\


\begin{equation}
\label{eq:WMSE}
\text{WMSE}=
\sum\nolimits_{(I,v,w) \in C_0} w (v-\bar{v_0})^2 + 
\sum\nolimits_{(I,v,w) \in C_1} w (v-\bar{v_0})^2
\end{equation}

\[\sum_{(I,v,w) \in C_0} w (v-\bar{v_0})^2=\sum_{i=1}^{\| C_0 \|} w_i (v_i - \bar{v_0})^2=
\sum_{i=1}^{\| C_0 \|} w_i (v_i^2 - 2v_i \bar{v_0} + \bar{v_0}^2)\]\\

Let $\| C_{0} \|=n$.\\

\[\sum_{i=1}^{n} w_i v_i^2 - 2\bar{v_0}\sum_{i=1}^{n} w_i v_i + 
\bar{v_0}^2\sum_{i=1}^{n} w_i=\]

Refer to \eqref{eq:WAGT0}:
\[
\sum_{i=1}^{\| C_0 \|}  w_i v_i = \bar{v_0} \cdot \sum_{i=1}^{\| C_0 \|}  w_i
\]

\[\sum_{i=1}^{n} w_i v_i^2 - 2\bar{v_0} (\bar{v_0} \cdot \sum_{i=1}^n w_i) + 
\bar{v_0}^2\sum_{i=1}^{n} w_i=\]

\[\sum_{i=1}^{n} w_i v_i^2 - 2\bar{v_0} \bar{v_0} \cdot \sum_{i=1}^n w_i 
+ \bar{v_0}^2\sum_i^n w_i=\]

\[\sum_{i=1}^{n} w_i v_i^2 - 2\bar{v_0}^2 \sum_{i=1}^n w_i  + 
\bar{v_0}^2 \sum_{i=1}^n w_i=\]

\[\sum_{i=1}^{n} w_i v_i^2 - \bar{v_0}^2 \sum_{i=1}^n w_i = \]
The ground truth, $v_i$, is $+1$ or $-1$, so ${v_i}^2=1$.\\

\begin{equation}
\label{eq:WMSE00}
\sum_{i=1}^{n} w_i - \bar{v_0}^2 \sum_{i=1}^n w_i =
(1 - \bar{v_0}^2) \cdot \sum_{i=1}^{n} w_i = WMSE0
\end{equation}

If $\sum_{i=1}^{n} w_i =1$ , \eqref{eq:WMSE00} will be:
\begin{equation}
\label{eq:WMSE0}
\text{WMSE0} = (1 - \bar{v_0}^2) \cdot \sum_{i=1}^{n} w_i = 1-\bar{v_0}^2
\end{equation}

\begin{equation}
\label{eq:WMSE1}
\text{WMSE1} = (1 - \bar{v_1}^2) \cdot \sum_{i=1}^{m} w_i = 1-\bar{v_1}^2
\end{equation}

\begin{equation}
\label{eq:WMSE_01}
\text{WMSE} = \text{WMSE1} + \text{WMSE2} = (1-\bar{v_0}^2) + (1-\bar{v_1}^2) 
\end{equation}\\

You are now using Gummi 0.6.1. Many new exciting features have been added to the 0.6 series. The document editor is now a tabbed instance, allowing multiple documents to be worked on simultaneously. Using the new projects menu, you can group files together for easy access. 

Support for two high-level {\LaTeX} building systems, \emph{rubber}\footnote{https://launchpad.net/rubber/} \& \emph{latexmk}\footnote{http://www.phys.psu.edu/{\textasciitilde}collins/software/latexmk-jcc/} has been added to this release as well. Your preferred typesetter can be configured through the Compilation tab in the Preferences menu. Typesetters that are not installed on your system will not be selectable. 

Added for your viewing convenience is a continuous preview mode for the PDF. This mode is enabled by default, but can also be disabled through the \emph{(View $\rightarrow$ Page layout in preview)} menu. Complementary to this feature is SyncTeX integration, which allows you to synchronize the position in your editor with the PDF preview. 

\section{Feedback}
We hope you will enjoy using this release as much as we enjoyed creating it. If you have comments, suggestions or wish to report an issue you are experiencing - contact us at: \emph{http://gummi.midnightcoding.org}.

\section{One more thing}
If you are wondering where your old default text is; it has been stored as a template. The template menu can be used to access and restore it. 

\end{document}
