%%%%%%%%%%%%%%%%%%%%%%%%%%%%%%%%%%%%%%%%%
% Journal Article
% LaTeX Template
% Version 1.3 (9/9/13)
%
% This template has been downloaded from:
% http://www.LaTeXTemplates.com
% http://en.wikibooks.org/wiki/LaTeX/Mathematics
% Original author:
% Frits Wenneker (http://www.howtotex.com)
%
% License:
% CC BY-NC-SA 3.0 (http://creativecommons.org/licenses/by-nc-sa/3.0/)
%
%%%%%%%%%%%%%%%%%%%%%%%%%%%%%%%%%%%%%%%%%

%----------------------------------------------------------------------------------------
%	PACKAGES AND OTHER DOCUMENT CONFIGURATIONS
%----------------------------------------------------------------------------------------

%\documentclass[twoside]{article}
\documentclass[a4paper,12pt]{article}

%\usepackage{lipsum} % Package to generate dummy text throughout this template
\usepackage{mathtools} % package fixes some amsmath quirks and adds some useful settings, symbols, and environments to amsmath
\usepackage[sc]{mathpazo} % Use the Palatino font
\usepackage[T1]{fontenc} % Use 8-bit encoding that has 256 glyphs
\linespread{1.05} % Line spacing - Palatino needs more space between lines
\usepackage{microtype} % Slightly tweak font spacing for aesthetics

\usepackage[hmarginratio=1:1,top=32mm,columnsep=20pt]{geometry} % Document margins
\usepackage{multicol} % Used for the two-column layout of the document
\usepackage[hang, small,labelfont=bf,up,textfont=it,up]{caption} % Custom captions under/above floats in tables or figures
\usepackage{booktabs} % Horizontal rules in tables
\usepackage{float} % Required for tables and figures in the multi-column environment - they need to be placed in specific locations with the [H] (e.g. \begin{table}[H])
\usepackage{hyperref} % For hyperlinks in the PDF

\usepackage{lettrine} % The lettrine is the first enlarged letter at the beginning of the text
\usepackage{paralist} % Used for the compactitem environment which makes bullet points with less space between them

\usepackage{cite}
%\usepackage[numbers,sort&compress]{natbib}
%\usepackage[style=numeric]{biblatex}
%\addbibresource{thomastsai.bib}
%\bibliographystyle{ieeetr}

\usepackage{abstract} % Allows abstract customization
\renewcommand{\abstractnamefont}{\normalfont\bfseries} % Set the "Abstract" text to bold
\renewcommand{\abstracttextfont}{\normalfont\small\itshape} % Set the abstract itself to small italic text

%http://en.wikibooks.org/wiki/LaTeX/Floats,_Figures_and_Captions
\usepackage{graphicx}
\usepackage{subcaption}
\graphicspath{ {dft/} }
\DeclareGraphicsExtensions{.pdf,.png,.jpg}

\usepackage{titlesec} % Allows customization of titles
\renewcommand\thesection{\Roman{section}} % Roman numerals for the sections
\renewcommand\thesubsection{\Roman{subsection}} % Roman numerals for subsections
\titleformat{\section}[block]{\large\scshape\centering}{\thesection.}{1em}{} % Change the look of the section titles
\titleformat{\subsection}[block]{\large}{\thesubsection.}{1em}{} % Change the look of the section titles

\usepackage{fancyhdr} % Headers and footers
\pagestyle{fancy} % All pages have headers and footers
\fancyhead{} % Blank out the default header
\fancyfoot{} % Blank out the default footer
%\fancyhead[C]{Running title $\bullet$ October 2014 $\bullet$ Vol. XXI, No. 1} % Custom header text
%\fancyfoot[RO,LE]{\thepage} % Custom footer text

%----------------------------------------------------------------------------------------
%	TITLE SECTION
%----------------------------------------------------------------------------------------

\title{\vspace{-15mm}\fontsize{24pt}{10pt}\selectfont\textbf{Study Report: Object Detection with Pixel Intensity Comparisons Organized in Decision Trees}} % Article title

\author{
\large
\textsc{Thomas Tsai}\thanks{A thank you or further information}\\[2mm] % Your name
\normalsize www.biotrump.com \\ % Your institution
\normalsize \href{mailto:thomas@biotrump.com}{thomas@biotrump.com} % Your email address
\vspace{-5mm}
}
\date{}

%----------------------------------------------------------------------------------------

\begin{document}

\maketitle % Insert title

\thispagestyle{fancy} % All pages have headers and footers

%----------------------------------------------------------------------------------------
%	ABSTRACT
%----------------------------------------------------------------------------------------

\begin{abstract}

%\noindent \lipsum[1] % Dummy abstract text
Pixel intensity comparison binary test. \cite{DBLP:journals/corr/abs-1305-4537} Decision Tree.

\end{abstract}

%----------------------------------------------------------------------------------------
%	ARTICLE CONTENTS
%----------------------------------------------------------------------------------------

\begin{multicols}{2} % Two-column layout throughout the main article text
\end{multicols}
%\onecolumn

\section{Introduction}

\lettrine[nindent=0em,lines=3]{I}\ am trying to list all the possible proofs.
%\lipsum[2-3] % Dummy text

\section{Weighted Mean Squared Error: WMSE}
A training data is a set $\{(I_s,v_s,w_s) : s=1,2,..,S\}$
where $v_s$ is the \textbf{ground truth value} associated with image $I_s$ and $w_s$ is its weight.
\\ A pixel intensity comparison binary test on a image I is defined as

\begin{equation}
\label{eq:bintest}
    bintest(I;l_1,l_2)= 
\begin{cases}
    0,		& \text{if } I(l_1)\leq I(l_2)\\
    1,      & \text{otherwise}
\end{cases}
\end{equation}\\
\\ A \textbf{bintest}, $f_i$, has \textbf{only two values: 0 or 1}, so a image will be classified as 0 or 1 by the \textbf{bintest}. If an image is classified to "0", it's in class/cluster $C_0$, otherwise it's in class/cluster $C_1$. \\
\\ Scalar $v_0$ and $v_1$ are \textbf{weighted averages of the ground truths} in $C_0$ and $C_1$, respectively.

\begin{equation}
\label{eq:WAGT0}
\bar{v_0}=\frac{\sum\nolimits_{(I,v,w) \in C_0} w_i v_i} { \sum\nolimits_{(I,v,w) \in C_0} w_i} =
\end{equation}
\[
\frac{\sum_{i=1}^{\| C_0 \|}  w_i v_i}
{\sum_{i=1}^{\| C_0 \|}  w_i}
\]
\begin{equation}
\label{eq:WAGT1}
\bar{v_1}=\frac{\sum\nolimits_{(I,v,w) \in C_1} w_i v_i} { \sum\nolimits_{(I,v,w) \in C_1} w_i}=
\end{equation}
\[
\frac{\sum_{i=1}^{\| C_1 \|}  w_i v_i}
{\sum_{i=1}^{\| C_1 \|}  w_i}
\]

The ground truth, $v_s$, of a positive sample is annotated with $+1$ and 
a negative sample is annotated with $-1$.\\


\begin{equation}
\label{eq:WMSE}
\text{WMSE}=
\sum\nolimits_{(I,v,w) \in C_0} w (v-\bar{v_0})^2 + 
\sum\nolimits_{(I,v,w) \in C_1} w (v-\bar{v_0})^2
\end{equation}

\[\sum_{(I,v,w) \in C_0} w (v-\bar{v_0})^2=\sum_{i=1}^{\| C_0 \|} w_i (v_i - \bar{v_0})^2=
\sum_{i=1}^{\| C_0 \|} w_i (v_i^2 - 2v_i \bar{v_0} + \bar{v_0}^2)\]\\

Let $\| C_{0} \|=n$.\\

\[\sum_{i=1}^{n} w_i v_i^2 - 2\bar{v_0}\sum_{i=1}^{n} w_i v_i + 
\bar{v_0}^2\sum_{i=1}^{n} w_i=\]

Refer to \eqref{eq:WAGT0}:
\[
\sum_{i=1}^{\| C_0 \|}  w_i v_i = \bar{v_0} \cdot \sum_{i=1}^{\| C_0 \|}  w_i
\]

\[\sum_{i=1}^{n} w_i v_i^2 - 2\bar{v_0} (\bar{v_0} \cdot \sum_{i=1}^n w_i) + 
\bar{v_0}^2\sum_{i=1}^{n} w_i=\]

\[\sum_{i=1}^{n} w_i v_i^2 - 2\bar{v_0} \bar{v_0} \cdot \sum_{i=1}^n w_i 
+ \bar{v_0}^2\sum_i^n w_i=\]

\[\sum_{i=1}^{n} w_i v_i^2 - 2\bar{v_0}^2 \sum_{i=1}^n w_i  + 
\bar{v_0}^2 \sum_{i=1}^n w_i=\]

\[\sum_{i=1}^{n} w_i v_i^2 - \bar{v_0}^2 \sum_{i=1}^n w_i = \]
The ground truth, $v_i$, is $+1$ or $-1$, so ${v_i}^2=1$.\\

\begin{equation}
\label{eq:WMSE00}
\sum_{i=1}^{n} w_i - \bar{v_0}^2 \sum_{i=1}^n w_i =
(1 - \bar{v_0}^2) \cdot \sum_{i=1}^{n} w_i = WMSE0
\end{equation}

If $\sum_{i=1}^{n} w_i =1$ , \eqref{eq:WMSE00} will be:
\begin{equation}
\label{eq:WMSE0}
\text{WMSE0} = (1 - \bar{v_0}^2) \cdot \sum_{i=1}^{n} w_i = 1-\bar{v_0}^2
\end{equation}

\begin{equation}
\label{eq:WMSE1}
\text{WMSE1} = (1 - \bar{v_1}^2) \cdot \sum_{i=1}^{m} w_i = 1-\bar{v_1}^2
\end{equation}

\begin{equation}
\label{eq:WMSE_01}
\text{WMSE} = \text{WMSE1} + \text{WMSE2} = (1-\bar{v_0}^2) + (1-\bar{v_1}^2) 
\end{equation}\\

%----------------------------------------------------------------------------------------
%	REFERENCE LIST
%----------------------------------------------------------------------------------------
%\printbibliography
%\bibliography{abbr_long,pubext}
\bibliography{thomastsai.bib}{}
\bibliographystyle{ieeetr}

%\begin{thebibliography}{99} % Bibliography - this is intentionally simple in this template

%\bibitem[Figueredo and Wolf, 2009]{Figueredo:2009dg}
%Figueredo, A.~J. and Wolf, P. S.~A. (2009).
%\newblock Assortative pairing and life history strategy - a cross-cultural
%  study.
%\newblock {\em Human Nature}, 20:317--330.
 
%\end{thebibliography}

%----------------------------------------------------------------------------------------

%\end{multicols}

\end{document}
