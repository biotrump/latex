%%%%%%%%%%%%%%%%%%%%%%%%%%%%%%%%%%%%%%%%%
% Journal Article
% LaTeX Template
% Version 1.3 (9/9/13)
%
% This template has been downloaded from:
% http://www.LaTeXTemplates.com
% http://en.wikibooks.org/wiki/LaTeX/Mathematics
% Original author:
% Frits Wenneker (http://www.howtotex.com)
%
% License:
% CC BY-NC-SA 3.0 (http://creativecommons.org/licenses/by-nc-sa/3.0/)
%
%%%%%%%%%%%%%%%%%%%%%%%%%%%%%%%%%%%%%%%%%

%----------------------------------------------------------------------------------------
%	PACKAGES AND OTHER DOCUMENT CONFIGURATIONS
%----------------------------------------------------------------------------------------

\documentclass[twoside]{article}

\usepackage{lipsum} % Package to generate dummy text throughout this template
\usepackage{mathtools} % package fixes some amsmath quirks and adds some useful settings, symbols, and environments to amsmath
\usepackage[sc]{mathpazo} % Use the Palatino font
\usepackage[T1]{fontenc} % Use 8-bit encoding that has 256 glyphs
\linespread{1.05} % Line spacing - Palatino needs more space between lines
\usepackage{microtype} % Slightly tweak font spacing for aesthetics

\usepackage[hmarginratio=1:1,top=32mm,columnsep=20pt]{geometry} % Document margins
\usepackage{multicol} % Used for the two-column layout of the document
\usepackage[hang, small,labelfont=bf,up,textfont=it,up]{caption} % Custom captions under/above floats in tables or figures
\usepackage{booktabs} % Horizontal rules in tables
\usepackage{float} % Required for tables and figures in the multi-column environment - they need to be placed in specific locations with the [H] (e.g. \begin{table}[H])
\usepackage{hyperref} % For hyperlinks in the PDF

\usepackage{lettrine} % The lettrine is the first enlarged letter at the beginning of the text
\usepackage{paralist} % Used for the compactitem environment which makes bullet points with less space between them

\usepackage{cite}
%\usepackage[numbers,sort&compress]{natbib}
%\usepackage[style=numeric]{biblatex}
%\addbibresource{thomastsai.bib}
%\bibliographystyle{ieeetr}

\usepackage{abstract} % Allows abstract customization
\renewcommand{\abstractnamefont}{\normalfont\bfseries} % Set the "Abstract" text to bold
\renewcommand{\abstracttextfont}{\normalfont\small\itshape} % Set the abstract itself to small italic text

\usepackage{titlesec} % Allows customization of titles
\renewcommand\thesection{\Roman{section}} % Roman numerals for the sections
\renewcommand\thesubsection{\Roman{subsection}} % Roman numerals for subsections
\titleformat{\section}[block]{\large\scshape\centering}{\thesection.}{1em}{} % Change the look of the section titles
\titleformat{\subsection}[block]{\large}{\thesubsection.}{1em}{} % Change the look of the section titles

\usepackage{fancyhdr} % Headers and footers
\pagestyle{fancy} % All pages have headers and footers
\fancyhead{} % Blank out the default header
\fancyfoot{} % Blank out the default footer
\fancyhead[C]{Running title $\bullet$ October 2014 $\bullet$ Vol. XXI, No. 1} % Custom header text
\fancyfoot[RO,LE]{\thepage} % Custom footer text

%----------------------------------------------------------------------------------------
%	TITLE SECTION
%----------------------------------------------------------------------------------------

\title{\vspace{-15mm}\fontsize{24pt}{10pt}\selectfont\textbf{ICA : Independent Component Analysis}} % Article title

\author{
\large
\textsc{Thomas Tsai}\thanks{A thank you or further information}\\[2mm] % Your name
\normalsize www.biotrump.com \\ % Your institution
\normalsize \href{mailto:thomas@biotrump.com}{thomas@biotrump.com} % Your email address
\vspace{-5mm}
}
\date{}

%----------------------------------------------------------------------------------------

\begin{document}

\maketitle % Insert title

\thispagestyle{fancy} % All pages have headers and footers

%----------------------------------------------------------------------------------------
%	ABSTRACT
%----------------------------------------------------------------------------------------

\begin{abstract}

%\noindent \lipsum[1] % Dummy abstract text
Independent component analysis, ICA, needs whitening the input signals.

\end{abstract}

%----------------------------------------------------------------------------------------
%	ARTICLE CONTENTS
%----------------------------------------------------------------------------------------

\begin{multicols}{2} % Two-column layout throughout the main article text

\section{Introduction}

\lettrine[nindent=0em,lines=3]{I}\ am trying to list all the possible proofs for ICA material.
%\lipsum[2-3] % Dummy text

%------------------------------------------------

\section{axioms}

\begin{compactitem}
\item If A is a n*n square matrix. Eigen value $\lambda$ and eigen vector $\hat{x}$: \cite{AntonELA10th}
\begin{equation}
\label{eq:eval}
A\hat{x}=\lambda \hat{x}
\end{equation}
The sign of $\lambda$ and $\hat{x}$ are not unique. And $\lambda$ and $\hat{x}$ are scalable.
\item A symmetric square matrix, $A=A^T$, can be eigen decomposition which is a special form of 
singular value decomposition, SVD, of a m*n matrix.
\begin{equation}
\label{eq:PDP}
A = PDP^{T}\ or\ U \Lambda U^T\
\end{equation}
where D or $\Lambda$ is a diagonal matrix whose diagonal components are eigen values.\\
\begin{equation}
\label{eq:diagm}
D=\Lambda= diag(\lambda_{ii}) = \begin{pmatrix}
       \lambda_{00}& 0 				& ..	&	0 	\\[0.3em]
       0 			& \lambda_{11} & ..		&	0 	\\[0.3em]
       .			& .				& ..	&	.	\\[0.3em]
       0 			& 0 			& ..	&	\lambda_{nn}\\[0.3em]
     \end{pmatrix}
\end{equation}

P or U is composed of orthonormal eigen vectors, so $P^{-1}=P^T$ or $U^{-1}=U^T$.\\
P or U is invertable and nonsigular. $det(P) \neq0$
\begin{equation}
\label{eq:pdpI}
I=P* P^{-1} = P* P^T=U* U^{-1} = U* U^T
\end{equation}

\item The diagonal matrix properties:\\
$det(D) = \prod_{i}^{} \lambda_{ii}$\\
\begin{equation}
\label{eq:pdpI}
D^{-1}=\Lambda^{-1}=diag(\lambda_{ii}^{-1})=
\begin{pmatrix}
       1/\lambda_{00}& 0 				& ..	&	0 	\\[0.3em]
       0 			& 1/\lambda_{11} & ..		&	0 	\\[0.3em]
       .			& .				& ..	&	.	\\[0.3em]
       0 			& 0 			& ..	&	1/\lambda_{nn}\\[0.3em]
     \end{pmatrix}
\end{equation}
\item standard deviation and variance:
\begin{equation}
\label{eq:std}
\sigma = \sqrt {\sum \frac{(x-\bar{x})^2}{N}}
\end{equation}
\begin{equation}
\label{eq:variance}
VAR(X)=\sigma^{2} = {\sum \frac{(x-\bar{x})^2}{N}}
\end{equation}

\item covariance: X and Y are two vectors, random variables. The covariance is a value.
\begin{equation}
\label{eq:covar}
Cov(X,Y)=\sigma(X,Y)=E[(X - E[X])(Y-E[Y])]
\end{equation}
\begin{equation}
\label{eq:covar1}
Cov(X,Y)=\frac{\sum(X_i-\bar{x})(Y_i-\bar{Y})}{N}=
\frac{\sum(x_i)(y_i)}{N}
\end{equation}
Deviation score $x_i, y_i$:
\begin{equation}
\label{eq:covar1}
x_{i}=(X_i-\bar{x}),\;
y_{i}=(Y_i-\bar{Y})
\end{equation}
Variance of X, $VAR(X)=Cov(X,X)=\sigma^{2}$, \ is a degenerated form of covariance.\\

\item covariance matrix : 
In probability theory and statistics, a covariance matrix (also known as dispersion matrix or variance–covariance matrix) 
is a matrix whose element in the i, j position is the covariance between the i th and j th elements of a random vector 
(that is, of a vector of random variables). Each element of the vector is a scalar random variable, 
either with a finite number of observed empirical values or with a finite or infinite number of 
potential values specified by a theoretical joint probability distribution of all the random variables.
\cite{wiki-covariance}

X is a square n*n symmetric matrix, so it's also a positive-definite matrix.

\begin{equation}
Cov[AX] = ACov[X]A^{T}\\\\
Y = \Lambda^{-1/2} U^{T} X,\\
let\ W = \Lambda^{-1/2} U^{T}.\ We\ call\ W\ "whitening"\ matrix.\\
Y = WX\\
\Lambda^{-1/2} = diag(1/\sqrt{\Lambda i i})\\\\
Cov[Y] = Cov[(\Lambda^{-1/2} U^{T}) X]=
(\Lambda^{-1/2} U^{T}) Cov[X] (\Lambda^{-1/2} U^{T}) ^T=\\\\
(\Lambda^{-1/2} U^{T})  U \Lambda U^{T} (\Lambda^{-1/2} U^{T}) ^T=
\Lambda^{-1/2} (U^{T}  U) \Lambda U^{T} (\Lambda^{-1/2} U^{T}) ^T=\\
\Lambda^{-1/2} (I) \Lambda U^{T} ({U^{T}}^T{\Lambda^{-1/2}}^T)=
\Lambda^{-1/2} \Lambda U^{T} (U{\Lambda^{-1/2}}^T)=
\Lambda^{-1/2} \Lambda (U^{T} U) {\Lambda^{-1/2}}^T=\\
\Lambda^{-1/2} \Lambda (I) {\Lambda^{-1/2}}^T=
\Lambda^{-1/2} \Lambda \Lambda^{-1/2}=\\
\end{equation}
\end{compactitem}

Maecenas sed ultricies felis. Sed imperdiet dictum arcu a egestas. 
\begin{compactitem}
\item Donec dolor arcu, rutrum id molestie in, viverra sed diam
\item Curabitur feugiat
\item turpis sed auctor facilisis
\item arcu eros accumsan lorem, at posuere mi diam sit amet tortor
\item Fusce fermentum, mi sit amet euismod rutrum
\item sem lorem molestie diam, iaculis aliquet sapien tortor non nisi
\item Pellentesque bibendum pretium aliquet
\end{compactitem}
\lipsum[4] % Dummy text

%------------------------------------------------

\section{Results}

\begin{table}[H]
\caption{Example table}
\centering
\begin{tabular}{llr}
\toprule
\multicolumn{2}{c}{Name} \\
\cmidrule(r){1-2}
First name & Last Name & Grade \\
\midrule
John & Doe & $7.5$ \\
Richard & Miles & $2$ \\
\bottomrule
\end{tabular}
\end{table}

\lipsum[5] % Dummy text

\begin{equation}
\label{eq:emc}
e = mc^2
\end{equation}

\lipsum[6] % Dummy text

%------------------------------------------------

\section{Discussion}

\subsection{Subsection One}

\lipsum[7] % Dummy text

\subsection{Subsection Two}

\lipsum[8] % Dummy text

%----------------------------------------------------------------------------------------
%	REFERENCE LIST
%----------------------------------------------------------------------------------------
%\printbibliography
%\bibliography{abbr_long,pubext}
\bibliography{thomastsai.bib}{}
\bibliographystyle{ieeetr}

%\begin{thebibliography}{99} % Bibliography - this is intentionally simple in this template

%\bibitem[Figueredo and Wolf, 2009]{Figueredo:2009dg}
%Figueredo, A.~J. and Wolf, P. S.~A. (2009).
%\newblock Assortative pairing and life history strategy - a cross-cultural
%  study.
%\newblock {\em Human Nature}, 20:317--330.
 
%\end{thebibliography}

%----------------------------------------------------------------------------------------

\end{multicols}

\end{document}
