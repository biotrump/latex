\documentclass[a4paper,11pt]{article}
\usepackage[latin1]{inputenc}
\usepackage{amsmath}
\usepackage[sloped]{fourier}
\usepackage{bm}
\usepackage[frenchb,english]{babel}
\renewcommand{\labelitemi}{\lefthand}
\setlength{\leftmargini}{1em}
\newcommand{\fourier}{Fourier-GUT\textit{enberg}}
\renewcommand{\ttdefault}{lmtt}
\title{\decofourleft{}\,Fourier-GUT{\itshape enberg}\,\decofourright}
\author{Michel Bovani\\\texttt{michel.bovani@wanadoo.fr}}
%%\renewcommand{\FrenchLabelItem}{\textbullet}
\begin{document}
\maketitle

\section{What is \fourier{}?}
\fourier{} is a \LaTeX{} typesetting system wich uses Adobe Utopia as its standard base font. 
Adobe Utopia has been choosen for several reasons. The main of them is that
four typefaces from the Utopia fonts packages have been gracefully donated to the X-consortium by 
Adobe. These typefaces (Utopia Regular, Utopia Italic, Utopia Bold, Utopia Bold Italic)
 are free of charges, and freely distributable (but it is \emph{not} free software: 
 see the licence in the read-me file!).
\par
Shortly, here are the main features of \fourier:
\begin{itemize}
\item \fourier{} provides all complementary typefaces needed to allow Utopia based \TeX{} typesetting. 
The system is absolutely stand-alone: apart from Utopia and fourier, no other typefaces are required.
\item \fourier{} provides two greeks, slanted and upright, that may be used in the same document.
\item It make it possible to typeset ``à la french'': upright roman uppercases, 
and upright greek in math mode.
\item \fourier{} do not use OT1 encoding at all. As in standard \LaTeX{} greek uppercases 
are in the text OT1 font, maths encodings have been redefined.
\item It is \emph{fully} T1 encoded: text symbols like ``dottlessj'' (\j, \textbf{\j}) or ``eng'' (\ng, \NG) are provided 
through a virtual fonts mechanism.
\item Optionnaly, the commercial Adobe expert complement may be fully used by fourier. 
It includes old-style digits, real (not faked) small caps, semi-bold, extra-black, etc.
 It may be usefull for professionnal typesetting, but of course, you have to buy the fonts!
\item The \verb=\boldmath= command is not still fully implemented, \emph{but} there are
 now bold versions of math letters fonts, which can be used with the 
 \item \fourier{} provides specific symbols, in math mode ($\llbracket, 
\rrbracket$, $\oiiint$) and in text mode (\texteuro, \eurologo, \aldineright).
\item There is a new package provided with \fourier{}: \texttt{fourier-orns}. 
This is for those who want only the \fourier{} logos \& decos, but not the \fourier{} fonts.
\textit{Please don't call it if you call \texttt{fourier}}.
\end{itemize}

\section{Installation \textsc{\&} setup}
The texmf tree provides a standard TDS.
You have to install all the \texttt{fourier} directories of the fourier texmf tree 
in one of yours texmf trees, according to your TDS specifications.

\begin{description}
\item[WARNING:] Note that in not up to date distributions, the \fourier{} map files
should be in 

\texttt{texmf/dvips/fourier}

and \emph{not} in

\texttt{texmf/fonts/map/dvips/fourier}
\end{description}
If you don't still have the four Utopia fonts, you have to install them too in

\texttt{texmf/fonts/type1/adobe/utopia/}

If you have a licence for the commercial Utopia packages, you have to 
rename the *.pfb files to suit the declarations in \texttt{fourier-utopia-expert.map} (or to 
modify this file). Mac fonts should be contverted to pfb format (with 
\texttt{t1unmac}, for instance).

You have now to setup your installation.
Depending, of the choosen texmf tree, it is possible that you have to regenerate 
first the database (\texttt{mktexlsr} command, for instance).

Then, if you have a recent \texttt{web2c} distribution 
(teTeX, TeXlive, fpTeX...), just run updmap.

\begin{description}
\item[UNIX:]\quad
\begin{verbatim}
    % updmap --enable Map fourier.map
\end{verbatim}
If you want to install the commercial complement too (remember that you will have to buy it...)
\begin{verbatim}
    % updmap --enable Map fourier-utopia-expert.map
\end{verbatim}
\item[Windows:]\quad
\begin{verbatim}
    % updmap --enable Map=fourier.map
\end{verbatim}
If you want to install the commercial complement too (remember that you will have to buy it...)
\begin{verbatim}
    % updmap --enable Map=fourier-utopia-expert.map
\end{verbatim}

\end{description}
 Please note that the setting
 of the dvi previewer is not documented here. On a \texttt{web2c} distribution, \texttt{updmap} should do it.

If you don't have \texttt{updmap} or if the syntax doens not match the described command, please tell me.

\section{Usage}
\subsection{Calling \fourier}
You call \fourier{} with: 
\begin{verbatim}
    \usepackage[<options>]{fourier}
\end{verbatim}
The options are:
\begin{enumerate}
\item \texttt{sloped} (default): in maths, lowercase greek is slanted, uppercase greek is upright, roman uppercase are slanted.
\[M\in\Gamma \iff OM=x\rho\]
\item \texttt{upright} (à la french): in maths, lowercase and uppercase greeks are upright, and so is roman uppercase.
\[\mathrm{M}\in\Gamma \iff \mathrm{OM}=x\otherrho\]
\item \texttt{widespace}: this option offers a larger interword space to those who think that the standard space of Utopia is too narrow\ldots
\item \texttt{expert}, \texttt{oldstyle}, \texttt{fulloldstyle}: in order to use these options you need the commercial complements of Utopia.
The \texttt{expert} option provides small caps (not faked), semi-bold, extra-black, (see the commands below) and more symbols in the TS1 companion encoding. The \texttt{oldstyle} option is the same, with oldstyle digits in text mode, and the \texttt{fulloldstyle} option is the same with oldstyle digits in text mode and in math mode.
\item \texttt{poorman} (default): if you don't have the commercial complement, you must use this option. The main disadvantage is that small caps will became \textsc{reduced Caps}.
\end{enumerate}
\subsection{Text commands}
First it is not usefull to call the T1 encoding (\verb=\usepackage[T1]{fontenc}=) 
because \verb=fourier= will do it anyway.

Note that the T1 encoding have been completed:
\begin{itemize}
\item \verb=\j=\quad \j, \textbf{\j}, \textit{\bfseries\j} etc. 
\item \verb=\ng=, \verb=\NG=\quad \ng, \NG, \textit{\ng}, 
\textbf{\NG} etc.
\item \verb=\textperthousand=, \verb=\textpertenthousand= \quad 
\textperthousand, \textpertenthousand, \textbf{\textperthousand}, 
\textbf{\itshape\textpertenthousand} etc.
\end{itemize}
\subsection{The companion encoding}
The  TS1 encoding is generally used through the 
 \verb=textcomp= package. This encoding is not fully implemented in \fourier{} 
 and the \verb=textcomp= package is called by \texttt{fourier}
 \par What is avaible is roughtly what is provided in the adobe standard encoding, with some complements:
\begin{itemize}
\item The euro symbol: \verb=\texteuro= \texteuro, 
\textit{\texteuro}, \textbf{\texteuro}, \textbf{\itshape\texteuro}.
%%\item A leaf: \verb=\textleaf= \textleaf
\end{itemize}
\par
\subsection{Fourier ornaments}
\fourier{} provides several logos and ornaments:
\begin{itemize}
\item A ``starred'' bullet: \verb=\starredbullet= \starredbullet
\item A variant of the euro symbol: \verb=\eurologo= \eurologo, \textbf{\eurologo}. 
Please note that the \verb=\textit= command will not change the slant of this symbol, but \verb=\textsl{\eurologo}= \textsl{\eurologo} will do it.
\item Decos and logos: \verb=\noway= \noway, \verb=\danger= \danger, \verb=\textxswup= \textxswup,
\verb=\textxswdown= \textxswdown, \verb=\decoone= \decoone, \verb=\decotwo= \decotwo,
\verb=\decothreeleft= \decothreeleft, \verb=\decothreeright= \decothreeright, 
\verb=\decofourleft= \decofourleft, \verb=\decofourright= \decofourright,\verb=\floweroneleft= \floweroneleft,
\verb=\floweroneright= \floweroneright,\verb=\lefthand= \lefthand, \verb=\righthand= \righthand, \verb=\decosix= \decosix, \verb=\bomb= \bomb.
\item Smileys: \verb=\grimace= \grimace, \verb=\textthing= \textthing.
\item Leaves: \verb=\leafleft= \leafleft, \verb=\leafright=\leafright, \verb=\leafNE= \leafNE, \verb=\aldineleft= \aldineleft,
\\ \verb=\aldineright= \aldineright, \verb=\aldine= \aldine, \verb=\aldinesmall= \aldinesmall.
\end{itemize}

Finally, some symbols are also provided in math mode, with other names:
\begin{itemize}
\item\verb=$\thething$= $\thething$ is a \emph{QEDsymbol} 
for a false proof. Of course, you don't need it!
\item\verb=$\xswordsup$=, \verb=$\xsworddown$= $\xswordsup$ may be used as tags for a debatted statement, or for anything else. $\xswordsdown$
\end{itemize}

\subsection{Mathematical encodings}
\subsection*{Compatibility with amsmath}
\fourier{} is compatible with the \texttt{amsmath} package, you no longer need to call 
\texttt{amsmath} \emph{before} \texttt{fourier} (thanks to Walter Schmidt).
The \texttt{amssymb} package will be usefull only if the wanted symbols does not still exists
 in \fourier{} (see the list below). If you finally need \texttt{amssymb}, 
 it is best to call it \emph{before} \texttt{fourier}.
\subsubsection*{Standard \LaTeX{} math commands} 
All standard \LaTeX{} math commands are supported by \fourier{}.

Of course, all these symbols have been redesigned in order to suit Utopia in 
terms of boldness, contrast and proportions. Greek is particularly concerned:

\vspace{-40pt}
{\Huge
\[a, \alpha, \mathrm{a}, \otheralpha, n, \eta, \mathrm{n}, \othereta, 
c, \epsilon, \varepsilon, \mathrm{c}, \otherepsilon, 
\othervarepsilon, \mathrm{A}, \Lambda\]\par\vspace{-12pt}

\vspace{-40pt}
\boldmath\bfseries
\[a, \alpha, \mathbf{a}, \otheralpha, n, \eta, \mathbf{n}, \othereta, 
c, \epsilon, \varepsilon, \mathbf{c}, \otherepsilon, 
\othervarepsilon, \mathbf{A}, \Lambda\]\par
}
\noindent but also delimiters (and plenty of others glyphs):
\def\testdelim#1#2{%
  - \left#1\left#1\left#1\left#1\left#1\left#1\left#1\left#1 
  \smash{\widetilde{D}} 
  \right#2\right#2\right#2\right#2\right#2\right#2\right#2\right#2 -}
\begingroup
\delimitershortfall-1pt
\begin{displaymath}
  \testdelim\{\} 
  \qquad 
  \testdelim()
\end{displaymath}
\begin{displaymath}
\testdelim\lfloor\rfloor
\qquad
\testdelim[]
\end{displaymath}
\endgroup
\subsubsection*{Mathematical alphabets}
Latin alphabets have been stolen to Utopia...

\begin{itemize}
\item Greek alphabet

\smallskip
Slanted version

\smallskip
$\alpha\,\beta\,\gamma\,\delta\,\epsilon\,\eta\,\zeta\,\theta\,\iota\,\kappa\,\lambda\,\mu\,\nu\,\xi\,\pi\,\rho\,\sigma\,\tau\,\upsilon\,\phi\,\chi\,\psi\,\omega$

$\otherGamma$\,$\otherDelta$\,$\otherTheta$\,$\otherLambda$\,$\otherXi$\,$\otherPi$\,$\otherSigma$\,$\otherUpsilon$\,$\otherPhi$\,$\otherPsi$\,$\otherOmega$


\smallskip
\textit{Variants:}
$\varepsilon$\,$\vartheta$\,$\varkappa$\,$\varpi$\,$\varvarpi$,\,$\varrho$\,$\varsigma$\,$\varphi$


\smallskip
Upright version

\smallskip
$\otheralpha\,\otherbeta\,\othergamma\,\otherdelta\,\otherepsilon\,\othereta\,\otherzeta\,\othertheta\,\otheriota\,\otherkappa\,\otherlambda\,\othermu\,\othernu\,\otherxi\,\otherpi\,\otherrho\,\othersigma\,\othertau\,\otherupsilon\,\otherphi\,\otherchi\,\otherpsi\,\otheromega$

$\Gamma$\,$\Delta$\,$\Theta$\,$\Lambda$\,$\Xi$\,$\Pi$\,$\Sigma$\,$\Upsilon$\,$\Phi$\,$\Psi$\,$\Omega$


\smallskip
\textit{Variants:}
$\othervarepsilon$\,$\othervartheta$\,$\othervarkappa$\,$\othervarpi$\,$\othervarvarpi$\,$\othervarrho$\,$\othervarsigma$\,$\othervarphi$



The way these symbols may be obtained depends of the required option (\texttt{sloped} ou \texttt{upright}). For instance, with
\begin{verbatim}
   \[\alpha,\otheralpha,\Omega,\otherOmega\]
\end{verbatim} 
You get
\[\alpha,\otheralpha,\Omega,\otherOmega\]
with the \texttt{sloped} option and 
\[\otheralpha,\alpha,\Omega,\otherOmega\]
with the \texttt{upright} option.

The \verb=\other= prefix allow you to switch from one greek to the other.

\smallskip

\item Calligraphic alphabet (\verb=\mathcal= command)

$\mathcal{ABCDEFGHIJKLMNOPQRSTUVWXYZ}$

\smallskip

\item Blackboard-bold alphabet (\verb=\mathbb= command). No need to load \texttt{amssymb} to get it!

$\mathbb{ABCDEFGHIJKLMNOPQRSTUVWXYZ1k}$
\end{itemize}
\subsubsection*{Provided \texttt{amssymb} commands}

\begin{tabular}{lll}
\verb=\leqslant= $\leqslant$&
\verb=\geqslant= $\geqslant$&
\verb=\blacktriangleleft= $\blacktriangleleft$\\
\verb=\intercal= $\intercal$&
\verb=\vDash= $\vDash$&
\verb=\blacktriangleright= $\blacktriangleright$\\
\verb=\nleqslant= $\nleqslant$&
\verb=\ngeqslant= $\ngeqslant$&
\verb=\nparallel= $\nparallel$\\
\verb=\complement= $\complement$&
\verb=\hslash= $\hslash$&
\verb=\hbar= $\hbar$\\
\verb=\nexists= $\nexists$&
\verb=\notowns= $\notowns$&
\verb=\varsubsetneq= $\varsubsetneq$\\
\verb=\smallsetminus= $\smallsetminus$&
\verb=\nvDash= $\nvDash$&
\verb=\square= $\square$\\
\verb=\leftleftarrows= $\leftleftarrows$&
\verb=\rightrightarrows= $\rightrightarrows$&
\verb=\subsetneqq= $\subsetneqq$\\
\verb=\curvearrowleft= $\curvearrowleft$&
\verb=\curvearrowright= $\curvearrowright$&
\verb=\blacksquare= $\blacksquare$
\end{tabular}
\subsubsection*{\fourier{} extended commands}
The \verb=\widehat= and \verb=\widetilde= commands have been extended (like in \texttt{yhmath}).
\[\widehat{x}\; \widehat{xx} \;\widehat{xxx} \;\widehat{xxxx} 
\;\widehat{xxxxx} \;\widehat{xxxxxx} \;\widetilde{x}\; \widetilde{xx} 
\;\widetilde{xxx} \;\widetilde{xxxx} \;\widetilde{xxxxx} 
\;\widetilde{xxxxxx} \]
\subsubsection*{\fourier{} specific commands}
The following commands are provided by \fourier{}.
\begin{itemize}
\item \verb=\varkappa=, \verb=\varvarrho=, \verb=\varvarpi=, \verb=\varpartialdiff= :
$\varkappa$, $\varvarrho$, $\varvarpi$, $\varpartialdiff$.
\item \verb=\parallelslant= et \verb=\nparallelslant= : 
$\parallelslant$, $\nparallelslant$.
\item \verb=\iint=, \verb=\iiint=,  
\verb=\oiint=,\verb=\oiiint=,\verb=\slashint= : 
$\iint,\iiint,\oiint,\oiiint,\slashint$
\[\iint,\iiint,\oiint,\oiiint,\slashint\]
\item \verb=\llbracket=, \verb=\rrbracket=, \verb=\VERT=
\begingroup
\delimitershortfall-1pt
\begin{displaymath}
  \testdelim\llbracket\rrbracket
  \qquad 
  \testdelim\VERT\VERT
  \end{displaymath}
  \endgroup
  Note that the first version of \fourier{} used \verb=\dblbrackleft= and \verb=\dblbrackright= in place of
  \verb=\llbracket= and \verb=\rrbracket=. The old commands still exist, but are deprecated.
 \item \verb=\wideparen= et \verb=\widering= (like in 
\texttt{yhmath}, but please note that it is necessary to call the \texttt{amsmath} package in order to get the \verb=\widering= command in \fourier).
 \[\wideparen{XXXXXXXXX}\quad \widering{(A\cup B)\cap(C\cup D)}\]
\item Finally \verb=\widearc= and \verb=\wideOarc=
\[\widearc{AMB}\quad \wideOarc{AMB}\]

\end{itemize}
\subsection{Usage of commercial typefaces}
The \texttt{expert}, \texttt{oldstyle} or \texttt{fulloldstyle} options, if usable, provides these complementary commands:
\begin{itemize}
\item  \verb=\textsb= \verb=\sbseries= semi-bold;
\item \verb=\textblack= \verb=\blackseries= extra-black;
\item \verb=\texttitle= \verb=\titleshape= titling 
(incomplete T1 encoding);
\item \verb=\oldstyle= to switch to the oldstyle digits with the \texttt{expert} option;
\item \verb=\lining= to switch to the lining digits with the \texttt{oldstyle} option.
\end{itemize}

\begin{center}\Huge
\decotwo
\end{center}

\end{document}