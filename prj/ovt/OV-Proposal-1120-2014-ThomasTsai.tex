\documentclass[a4paper,10pt]{report}
%\documentclass[a4paper,12pt]{article}
\usepackage[utf8]{inputenc}

%----------------------------------------------------------------------------------------
%	PACKAGES AND OTHER DOCUMENT CONFIGURATIONS
%----------------------------------------------------------------------------------------

%\documentclass[twoside]{article}
%\documentclass[a4paper,12pt]{article}

%\usepackage{lipsum} % Package to generate dummy text throughout this template
\usepackage{mathtools} % package fixes some amsmath quirks and adds some useful settings, symbols, and environments to amsmath
\usepackage[sc]{mathpazo} % Use the Palatino font
\usepackage[T1]{fontenc} % Use 8-bit encoding that has 256 glyphs
\linespread{1.05} % Line spacing - Palatino needs more space between lines
\usepackage{microtype} % Slightly tweak font spacing for aesthetics

\usepackage[hmarginratio=1:1,top=32mm,columnsep=20pt]{geometry} % Document margins
\usepackage{multicol} % Used for the two-column layout of the document
\usepackage[hang, small,labelfont=bf,up,textfont=it,up]{caption} % Custom captions under/above floats in tables or figures
\usepackage{booktabs} % Horizontal rules in tables
\usepackage{float} % Required for tables and figures in the multi-column environment - they need to be placed in specific locations with the [H] (e.g. \begin{table}[H])
\usepackage{hyperref} % For hyperlinks in the PDF

\usepackage{lettrine} % The lettrine is the first enlarged letter at the beginning of the text
\usepackage{paralist} % Used for the compactitem environment which makes bullet points with less space between them
\usepackage{url}
\usepackage{cite}
%\usepackage[numbers,sort&compress]{natbib}
%\usepackage[style=numeric]{biblatex}
%\addbibresource{thomastsai.bib}
%\bibliographystyle{ieeetr}

\usepackage{abstract} % Allows abstract customization
\renewcommand{\abstractnamefont}{\normalfont\bfseries} % Set the "Abstract" text to bold
\renewcommand{\abstracttextfont}{\normalfont\small\itshape} % Set the abstract itself to small italic text

%http://en.wikibooks.org/wiki/LaTeX/Floats,_Figures_and_Captions
\usepackage{graphicx}
\usepackage{subcaption}
\graphicspath{ {dft/} }
\DeclareGraphicsExtensions{.pdf,.png,.jpg}

\usepackage{titlesec} % Allows customization of titles
\renewcommand\thesection{\Roman{section}} % Roman numerals for the sections
\renewcommand\thesubsection{\Roman{subsection}} % Roman numerals for subsections
\titleformat{\section}[block]{\large\scshape\centering}{\thesection.}{1em}{} % Change the look of the section titles
\titleformat{\subsection}[block]{\large}{\thesubsection.}{1em}{} % Change the look of the section titles

\usepackage{fancyhdr} % Headers and footers
\pagestyle{fancy} % All pages have headers and footers
\fancyhead{} % Blank out the default header
\fancyfoot{} % Blank out the default footer
\fancyhead[C]{Camera Sensor Applications $\bullet$ November 2014} % Custom header text $\bullet$ Vol. XXI, No. 1
\fancyfoot[RO,L]{\thepage} % Custom footer text
% Title Page
%----------------------------------------------------------------------------------------
%	TITLE SECTION
%----------------------------------------------------------------------------------------

\title{\vspace{-15mm}\fontsize{24pt}{10pt}\selectfont\textbf{Two Applications of Camera Sensors with Computer Vision}} % Article title
\author{
\large
\textsc{Thomas Tsai}\\[2mm] % Your name, \thanks{}
\normalsize www.biotrump.com \\ % Your institution
\normalsize \href{mailto:thomas@biotrump.com}{thomas@biotrump.com} % Your email address
\vspace{-5mm}
}
\date{}
\setlength{\headheight}{15pt}

%\let\myBib\thebibliography
%\let\endmyBib\endthebibliography
%\renewcommand\thebibliography[1]{\ifx\relax#1\relax\else\myBib{#1}\fi}

\setcounter{secnumdepth}{5} % seting level of numbering (default for "report" is 3). With ''-1'' you have non number also for chapters
%\setcounter{tocdepth}{5} % if you want all the levels in your table of contents

\begin{document}
\maketitle
%\thispagestyle{fancy} % All pages have headers and footers

%----------------------------------------------------------------------------------------
%	ABSTRACT
%----------------------------------------------------------------------------------------
\begin{abstract}
Two Applications are proposed. One is baby monitor and the other is smart glass.

\end{abstract}

\chapter{Baby Monitor}
\section {background}
\textbf{Sudden infant death syndrome (SIDS)} also known as \textbf{cot death} or
\textbf{crib death} is the sudden death of an infant that is not predicted by
medical history and remains unexplained after a thorough forensic autopsy and
detailed death scene investigation. Infants are at the \textbf{highest risk for
SIDS during sleep}. Typically the infant is found dead after having been put
to bed, and exhibits \textbf{no signs of having struggled}.\cite{wiki-SIDS}\\

\begin{figure}[h]
  \centering
	\includegraphics[width=0.7\textwidth, keepaspectratio=true]{baby-sids-prevention}
  \caption{SIDS Preventions}\cite{SIDS-healthystartorange}
  \label{fig:sids}
\end{figure}

\section {The solutions}
\label {TheSolutions}
Refer to \cite{Poh2011Advancements}\cite{Wu12Eulerian}, we develop the baby monitor camera
with computer vision to provide the following features:
\begin{figure}[h]
  \centering
	\includegraphics[width=0.7\textwidth, keepaspectratio=true]{mit-vidmag-teaser}
  \caption{Mit researches for hear beat detection}\cite{Wu12Eulerian}\cite{midvidmag-youtube}
  \label{fig:mit-vidmag}
\end{figure}

\begin{figure}[h]
  \centering
	\includegraphics[width=0.8\textwidth, keepaspectratio=true]{baby-pr}
  \caption{baby pulse rate and respiratory rate}\cite{Hao-Yu-2013}
  \label{fig:baby-pr}
\end{figure}

\begin{compactitem}
\item motion detection
\end{compactitem}
\begin{compactitem}
\item pulse rate detection
\end{compactitem}
\begin{compactitem}
\item respiratory rate detection
\end{compactitem}
\begin{compactitem}
\item foreign object detection
\end{compactitem}
\begin{compactitem}
\item expression recognition
\end{compactitem}
\begin{compactitem}
\item audio/voice recognition
\end{compactitem}
\begin{compactitem}
\item room temperature and air humidity monitor
\end{compactitem}
\begin{compactitem}
\item ambient light strength monitor
\end{compactitem}

\section{The baby monitor camera}
In order to implement the functions listed in section \ref{TheSolutions}, the baby monitor needs some cmos sensors:
\begin{compactitem}
\item RGB cmos sensor
\end{compactitem}
\begin{compactitem}
\item IR CMOS sensor
\end{compactitem}
\begin{compactitem}
\item Ambient Light Sensor
\end{compactitem}
\begin{compactitem}
\item LED Light Source
\end{compactitem}
\begin{compactitem}
\item InfraRed LED Light Source
\end{compactitem}

\begin{figure}[ht]
  \centering
	\includegraphics[width=0.7\textwidth, keepaspectratio=true]{paby-stand}
  \caption{The set of baby monitor }\cite{paby-monitor}
  \label{fig:paby-stand}
\end{figure}

\begin{figure}[ht]
  \centering
	\includegraphics[width=0.7\textwidth, keepaspectratio=true]{paby-camera}
  \caption{The camera of the baby monitor}\cite{paby-monitor}
  \label{fig:paby-camera}
\end{figure}


%Figure~\ref{fig:paby-camera} shows a photograph of a gull.\cite{paby-monitor}
\clearpage
\chapter{Smart Glass}
We will focus on industrial and medical smart glass development with different light spectra.

\begin{figure}[ht]
  \centering
	\includegraphics[width=0.7\textwidth, keepaspectratio=true]{medical-loupe-1}
  \caption{medical loupe}\cite{njnorth}
  \label{fig:medical-loupe}
\end{figure}

%Figure~\ref{fig:medical-loupe} shows a photograph of a gull.\cite{njnorth}

%----------------------------------------------------------------------------------------
%	REFERENCE LIST
%----------------------------------------------------------------------------------------
\clearpage
%\printbibliography
%\bibliography{abbr_long,pubext}
\bibliography{thomastsai.bib}{}
\bibliographystyle{ieeetr}
%\bibliographystyle{plain}

%\begin{thebibliography}{99} % Bibliography - this is intentionally simple in this template

%\bibitem[Figueredo and Wolf, 2009]{Figueredo:2009dg}
%Figueredo, A.~J. and Wolf, P. S.~A. (2009).
%\newblock Assortative pairing and life history strategy - a cross-cultural
%  study.
%\newblock {\em Human Nature}, 20:317--330.

%\end{thebibliography}

%----------------------------------------------------------------------------------------

\end{document}
