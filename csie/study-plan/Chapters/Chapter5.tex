% Chapter 5
\chapter{Photoplethysmogram} % Main chapter title

\label{Chapter5} % For referencing the chapter elsewhere, use \ref{Chapter1}

\lhead{Chapter 5. \emph{Photoplethysmogram}} % This is for the header on each page - perhaps a shortened title

%----------------------------------------------------------------------------------------
\section{Photoplethysmogram}
\cite{wiki-PPG}
A photoplethysmogram (PPG) is an optically obtained plethysmogram,
a volumetric measurement of an organ.
A PPG is often obtained by using a pulse oximeter
which illuminates the skin and measures changes in light absorption.[1]
A conventional pulse oximeter monitors the perfusion of blood to the dermis
and subcutaneous tissue of the skin.

Diagram of the layers of human skin With each cardiac cycle the heart pumps
blood to the periphery. Even though this pressure pulse is somewhat damped
by the time it reaches the skin, it is enough to distend the arteries and
arterioles in the subcutaneous tissue.
If the pulse oximeter is attached without compressing the skin,
a pressure pulse can also be seen from the venous plexus, as a small secondary peak.

The change in volume caused by the pressure pulse is detected by illuminating the skin
with the light from a light-emitting diode (LED) and then measuring the amount of
light either transmitted or reflected to a photodiode.
Each cardiac cycle appears as a peak, as seen in the figure.
Because blood flow to the skin can be modulated by multiple other physiological systems,
the PPG can also be used to monitor breathing, hypovolemia,
and other circulatory conditions.[2] Additionally, the shape of the PPG waveform
differs from subject to subject, and varies with the location and
manner in which the pulse oximeter is attached.

Sites for measuring PPG
While pulse oximeters are a commonly used medical device the PPG derived from them is
rarely displayed, and is nominally only processed to determine heart rate.
PPGs can be obtained from transmissive absorption (as at the finger tip) or
reflection (as on the forehead).

In outpatient settings, pulse oximeters are commonly worn on the finger.
However, in cases of shock, hypothermia, etc. blood flow to the periphery
can be reduced, resulting in a PPG without a discernible cardiac pulse.
In this case, a PPG can be obtained from a pulse oximeter on the head,
with the most common sites being the ear, nasal septum, and forehead.

PPGs can also be obtained from the vagina and esophagus.

Motion artifacts have been shown to be a limiting factor preventing accurate readings
during exercise and free living conditions.
January 2010, a group at Qualcomm Inc. (H. Garudadri, P. K. Bahedi, S. Majumdar)
filed a US patent application describing a so-called body area network with,
among others, a finger PPG and a 3D acceleroter. Sampling parameters,
in particular the under sampling ratio of the used compressed sampling are adjusted based
on sensor readings. [3]

\section{Remote PPG}

\begin{compactitem}

\item MIT EVM
\end{compactitem}

%----------------------------------------------------------------------------------------

\section{In Closing}

You have reached the end of this mini-guide. You can now rename or overwrite this pdf file and begin writing your own `\texttt{Chapter1.tex}' and the rest of your thesis. The easy work of setting up the structure and framework has been taken care of for you. It's now your job to fill it out!

Good luck and have lots of fun!
