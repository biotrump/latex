% Chapter 2
\chapter{Fundamental Mathematics} % Main chapter title

\label{Chapter2} % For referencing the chapter elsewhere, use \ref{Chapter1} 

\lhead{Chapter 2. \emph{Fundamental Mathematics}} % This is for the header on each page - perhaps a shortened title

%----------------------------------------------------------------------------------------
\section{Linear algebra}
\begin{compactitem}

\item Diagonal Matrix Properties
In linear algebra, a diagonal matrix is a matrix (usually a square matrix) in which 
the entries outside the main diagonal are all zero. The diagonal entries themselves 
may or may not be zero. Thus, the matrix $D = (d_{i,j})$ with n columns and n rows is diagonal 
if:
$d_{i,j}=0,\, if \,\, i \neq j \,\, \forall i,j \in \{1,2,3,...n\} $
\\For example, the following matrix is diagonal:\\
$\begin{pmatrix}
       1	&0 	&0	\\[0.3em]
       0 	&4 	&0 	\\[0.3em]
       0 	& 0 	&-2	\\[0.3em]
\end{pmatrix}
$ \cite{wiki-Diagonal_matrix}
\\
\\The general form :$D=diag(\lambda_{ii}) =$
\begin{equation}
\label{eq:diagm}
\begin{pmatrix}
       \lambda_{00}& 0 				& ..	&	0 	\\[0.3em]
       0 			& \lambda_{11} & ..		&	0 	\\[0.3em]
       .			& .				& ..	&	.	\\[0.3em]
       0 			& 0 			& ..	&	\lambda_{nn}\\[0.3em]
     \end{pmatrix}
\end{equation}

\begin{equation}
\label{eq:pdpD}
D^{-1}=diag(\lambda_{ii}^{-1})=
\begin{pmatrix}
       1/\lambda_{00}& 0 				& ..	&	0 	\\[0.3em]
       0 			& 1/\lambda_{11} & ..		&	0 	\\[0.3em]
       .			& .				& ..	&	.	\\[0.3em]
       0 			& 0 			& ..	&	1/\lambda_{nn}\\[0.3em]
\end{pmatrix}
, \forall \Lambda_{ii} \neq 0,
\end{equation}\\

The determinant of a Diagonal matrix is the product of all the diagonal elements.
\begin{equation}
\label{eq:detD}
det(D) = \prod_{i}^{} \lambda_{ii}
\end{equation}

\begin{equation}
\label{eq:det-pdpD}
\\
det(D^{-1})=
\prod_{i}^{} \frac{1}{\lambda_{ii} }\\
=\prod_{i}^{} \lambda_{ii}^{-1} \\
\end{equation}\\

%----------------------------------------------------------------------------------------
\item {Eigen Values and Eigen Vectors}
\\If A is a n*n square matrix. Eigen value $\lambda$ and \textbf{eigen} vector $\hat{x}$: \cite{AntonELA10th}
\begin{equation}
\label{eq:eval}
A\hat{x}=\lambda \hat{x}
\end{equation}
The sign of $\lambda$ and $\hat{x}$ are not unique. And $\lambda$ and $\hat{x}$ are scalable.
\\$ (A-\lambda I)X=0$
Set the determinant to zero to obtain the polynomial equation to solve the eigen values.
$det(A-\lambda I)=0$

\item singular value decomposition
\\U and V are orthogonal matrix, $U^{-1}=U^{\top}, V^{-1}=V^{\top}$\\
r=rank A
\begin{flalign}
\underbrace{\mathbf{A}}_{M \times N} = \underbrace{\mathbf{U}}_{M \times M} \times 
\underbrace{\mathbf{\Sigma}}_{M\times N} \times 
\underbrace{\mathbf{V}^{\text{T}}}_{N \times N} = &
\end{flalign}

\hspace*{-3cm}\vbox{
\begin{flalign}
\begin{tikzpicture}[
baseline,
mymat/.style={
  matrix of math nodes,
  ampersand replacement=\&,
  left delimiter=(,
  right delimiter=),
  nodes in empty cells,
  nodes={outer sep=-\pgflinewidth,text depth=0.5ex,text height=2ex,text width=1.2em}
}
]
\begin{scope}[every right delimiter/.style={xshift=-3ex}]
\matrix[mymat] (matu)
{
 \& \& \& \& \& \\
\& \& \& \& \& \\
\& \& \& \& \& \\
\& \& \& \& \& \\
\& \& \& \& \& \\
\& \& \& \& \& \\
};
\node 
  at ([shift={(3pt,-7pt)}]matu-3-2.west) 
  {$\cdots$};
\node 
  at ([shift={(3pt,-7pt)}]matu-3-5.west) 
  {$\cdots$};
\foreach \Columna/\Valor in {1/1,3/r,4/{r+1},6/m}
{
\draw 
  (matu-1-\Columna.north west)
    rectangle
  ([xshift=4pt]matu-6-\Columna.south west);
\node[above] 
  at ([xshift=2pt]matu-1-\Columna.north west) 
  {$u_{\Valor}$};
}
\draw[decorate,decoration={brace,mirror,raise=3pt}] 
  (matu-6-1.south west) -- 
   node[below=4pt] {$\Mcol(A)$}
  ([xshift=4pt]matu-6-3.south west);
\draw[decorate,decoration={brace,mirror,raise=3pt}] 
  (matu-6-4.south west) -- 
   node[below=4pt] {$\Mnull(A)$}
  ([xshift=4pt]matu-6-6.south west);
\end{scope}
\matrix[mymat,right=10pt of matu] (matsigma)
{
\sigma_{1} \& \& \& \& \& \\
\& \ddots \& \& \& \& \\
\& \& \sigma_{r} \& \& \& \\
\& \& \& 0 \& \& \\
\& \& \& \& \ddots \& \\
\& \& \& \& \& 0 \\
};
%\begin{scope}[every right delimiter/.style={xshift=-3ex}]
\matrix[mymat,right=25pt of matsigma] (matv)
{
 \& \& \& \& \& \\
\& \& \& \& \& \\
\& \& \& \& \& \\
\& \& \& \& \& \\
\& \& \& \& \& \\
\& \& \& \& \& \\
};
\foreach \Fila/\Valor in {1/1,3/r,4/{r+1},6/n}
{
\draw 
  ([yshift=-6pt]matv-\Fila-1.north west)
    rectangle
  ([yshift=-10pt]matv-\Fila-6.north east);
\node[right=12pt] 
  at ([yshift=-8pt]matv-\Fila-6.north east) 
  {$v^{T}_{\Valor}$};
}
\draw[decorate,decoration={brace,raise=37pt}] 
  ([yshift=-6pt]matv-1-6.north east) -- 
   node[right=38pt] {$\Mrow(A)$}
  ([yshift=-10pt]matv-3-6.north east);
\draw[decorate,decoration={brace,raise=37pt}] 
  ([yshift=-6pt]matv-4-6.north east) -- 
   node[right=38pt] {$\Mnull(A)$}
  ([yshift=-10pt]matv-6-6.north east);
\end{tikzpicture}
\end{flalign}}

%----------------------------------------------------------------------------------------
\item Symmetric Matrix:\\
A symmetric matrix is a square matrix that is equal to its transpose. 
Formally, matrix A is symmetric if $A = A^{\top}$.
So if the entries are written as $A = (a_{ij}), \,then\,\, a_{ij} = a_{ji}$.
\\
A real sqaure symmetric matrix can be Eigen Value Decomposition : $A=PDP^{\top}$.
D is a diagonal matrix whose elements are eigen values of A.
P is composed of orthogonal eigen vectors, so $P^{-1}=P^{\top}$.\\
\begin{equation}
\label{eq:pdpI}
I=P* P^{-1} = P* P^{\top}
\end{equation}

\item positive definite matrix : $x^{\top} A x \geq 0 ,\forall\, x \, \in \, \mathbb{R}^{N}$
\\
Every positive definite matrix has positive eigenvalues.

 
\end{compactitem}
%----------------------------------------------------------------------------------------

\section{Covariance and Whitening}
\begin{compactitem}
 
\item {standard deviation and variance}
\begin{equation}
\label{eq:std}
\sigma = \sqrt {\sum \frac{(x-\bar{x})^2}{N}}
\end{equation}
\begin{equation}
\label{eq:variance}
VAR(X)=\sigma^{2} = {\sum \frac{(x-\bar{x})^2}{N}}
\end{equation}

\item covariance: X and Y are two vectors, random variables. The covariance is a value.
\begin{equation}
\label{eq:covar}
Cov(X,Y)=\sigma(X,Y)=E[(X - E[X])(Y-E[Y])]
\end{equation}
\begin{equation}
\label{eq:covar1}
Cov(X,Y)=\frac{\sum(X_i-\bar{X})(Y_i-\bar{Y})}{N}=
\frac{\sum(x_i)(y_i)}{N}
\end{equation}
Deviation score $x_i, y_i$:
\begin{equation}
\label{eq:covar1}
x_{i}=(X_i-\bar{X}),\;
y_{i}=(Y_i-\bar{Y})
\end{equation}
Variance of X, $VAR(X)=Cov(X,X)=\sigma^{2}$, \ is a degenerated form of covariance.\\

\item covariance matrix: is a measure of the extent to which corresponding elements from two sets
of ordered data move in the same direction. We use the following formula to compute covariance.
\cite{wiki-covariance}\cite{STAT-covariance}\\
\\Let $X_i$ be the \textbf{column} vector in a Matrix \textbf{X}=
$\begin{pmatrix} X_1 & X_2 &... & X_c \end{pmatrix}$\\
The covariance matrix of \textbf{X} is:\\
$M_{cxc} = \Sigma{(\textbf{X})}=$
\[
\begin{pmatrix}
       COV(X_1,X_1) 	& COV(X_1,X_2) 	& ...	& COV(X_1,X_c)	\\[0.3em]
       COV(X_2,X_1) 	& COV(X_2,X_2) 	& ...	& COV(X_2,X_c)	\\[0.3em]
       ...		& ...			& ...	&	...		\\[0.3em]
       COV(X_c,X_1) 	& COV(X_c,X_2) 	& ...	& COV(X_c,X_c)	\\[0.3em]
\end{pmatrix}
=\]
\begin{equation}
\label{eq:covarm1}
\begin{pmatrix}
       \sum x_{1i}^{2}/N 	& \sum x_{1i} x_{2i}/N 	& ...	& \sum x_{1i} x_{ci}/N	\\[0.3em]
       \sum x_{2i} x_{1i}/N 	& \sum x_{2i} x_{2i}/N 	& ...	& \sum x_{2i} x_{ci}/N	\\[0.3em]
       ...		& ...			& ...	&	...		\\[0.3em]
       \sum x_{ci} x_{1i}/N 	& \sum x_{ci} x_{2i}/N 	& ...	& \sum x_{ci}^{2}/N	\\[0.3em]
\end{pmatrix}
=(1/N)
\begin{pmatrix}
       \sum x_{1i}^{2} 	& \sum x_{1i} x_{2i} 	& ...	& \sum x_{1i} x_{ci}	\\[0.3em]
       \sum x_{2i} x_{1i} 	& \sum x_{2i} x_{2i} 	& ...	& \sum x_{2i} x_{ci}	\\[0.3em]
       ...		& ...			& ...	&	...		\\[0.3em]
       \sum x_{ci} x_{1i} 	& \sum x_{ci} x_{2i} 	& ...	& \sum x_{ci}^{2}	\\[0.3em]
\end{pmatrix}
\end{equation}
where\\
N is the dim of the column vector $X_i$.\\
$x_i$ is a deviation score from the ith data set.\\
$\sum x_i^2 / N$ is the variance of elements from the ith data set.\\
$\sum x_i x_j / N$ is the covariance for elements from the ith and jth data sets.\\

\item Properties of Covariance matrix : 
\textbf{square, symmetric and positive semidefinite}: \\
Eigen value decomposition for a \textbf{square, symmetric} matrix: 
\begin{equation}
\label{eq:covd}
\Sigma(\textbf{X}) = U\Lambda U^T
\end{equation}


When A is positive semi-definite, $x^T A x >= 0$ and all eigen values are positive.  $\Sigma(\textbf{X})$ has all positive eigen values.

Diagonal matrix $\Lambda$ has all positive eigen values in its diagonal elements. $\Lambda ^ {1/2}$ doest exist and $\Lambda ^ {-1/2}$ doest exist, too.\\

\textbf{linearity of expectation matrix}: Let \textbf{X} be a random vector with covariance matrix
$\Sigma(\textbf{X})$, and let A be a matrix that can act on \textbf{X}. The covariance matrix of the vector A\textbf{X} is:
\begin{equation}
\label{eq:covlinear}
\Sigma(A\textbf{X}) = A\Sigma(\textbf{X})A^{T}
\end{equation}

\item whiten: Refer to \eqref{eq:covd},  A \textbf{whitening} matrix W is defined as
\begin{equation}
\label{eq:whitem}
W=U\Lambda^{-1/2} U^{T}
\end{equation}
This W is not unique, because $\Lambda$ and $U^T$ are not unique.
When the diagonal elements, eigen value, of the diagonal matrix, $\Lambda$, are changed in the order.
$U$ is also changed .

Let X and Y are \textbf{row-major} matrix, i.e, signals are in row vectors.
\begin{equation}
\label{eq:whitened}
Y_{nm} = W_{nn}*\textbf{X}_{nm}= (U\Lambda^{-1/2} U^{T}) \textbf{X}_{nm}
\end{equation}
Sometimes, column-major is used in fortran and some math tools,
\begin{equation}
\label{eq:whitened-col}
Y' = (W_{nn}*\textbf{X}_{nm})' = \textbf{X}'*W'= \textbf{X}*(U\Lambda^{-1/2} U^{T})'=\textbf{X}*(U^{T}\Lambda^{-1/2} U) 
\end{equation}
be a \textbf{whitened matrix} of \textbf{X}.\\
The covariance matrix of Y which is a whitened matrix of X.
\[
\Sigma (Y)=\Sigma (U\Lambda^{-1/2} U^{T}) \textbf{X}
\]
Refer to the equation \eqref{eq:covlinear}
\[
A=(U\Lambda^{-1/2} U^{T})
\]
so
\[
\Sigma (Y)=\Sigma (U\Lambda^{-1/2} U^{T}) \textbf{X}=
(U\Lambda^{-1/2} U^{T})(\Sigma (\textbf{X})) (U\Lambda^{-1/2} U^{T})^{T}
\]
Refer to the equation \eqref{eq:covd}\\
$=(U\Lambda^{-1/2} U^{T}) (U \Lambda U^{T}) (U\Lambda^{-1/2} U^{T}) ^T$\\
$=U\Lambda^{-1/2} (U^{T}  U) \Lambda U^{T} (U\Lambda^{-1/2} U^{T}) ^T$\\
Refer to the equation \eqref{eq:pdpI}\\
$=U\Lambda^{-1/2} (I) \Lambda U^{T} ({U^{T}}^T{\Lambda^{-1/2}}^T U^T)$\\
$=U\Lambda^{-1/2} \Lambda U^{T} (U{\Lambda^{-1/2}}^T) U^T$\\
$=U\Lambda^{-1/2} \Lambda (U^{T} U) {\Lambda^{-1/2}}^T U^T$\\
$=U\Lambda^{-1/2} \Lambda (I) {\Lambda^{-1/2}}^T U^T$\\
$=U\Lambda^{-1/2} \Lambda \Lambda^{-1/2} U^T$\\
$=U\Lambda^{(-1/2) + (1) + (-1/2)} U^T$\\
$=U I U^T = U U^T = I $\\
So \textbf{Y is whitened}, because its covariance matrix is an Identity matrix.
All row vectors in Y has no correlation!

\item \textbf{Simplification} for a \textbf{normalized} vector $\hat{X_i}$:\\
Let \textbf{X}=$\begin{pmatrix} X_1 & X_2 & ... & X_c \end{pmatrix}, X_1,X_2,X_c$ are column vectors which have been normalized.
(zero mean )\\

Refer to the equation \eqref{eq:covar1}, Deviation score $x_k, y_k$:\\
$x_{k}=(X_{ik}-\bar{X_i})=(X_{ik} - 0)= X_{ik}$ \\
$y_{k}=(Y_{ik}-\bar{Y_i})=(Y_{ik} - 0) = Y_{ik}$


Refer to the equation \eqref{eq:covarm1}, covariance matrix $\Sigma(\textbf{X})$=
\[
(1/N)
\begin{pmatrix}
       \sum x_{1k}^{2} 	& \sum x_{1k} x_{2k} 	& ...	& \sum x_{1k} x_{ck}	\\[0.3em]
       \sum x_{2k} x_{1k} 	& \sum x_{2k} x_{2k} 	& ...	& \sum x_{2k} x_{ck}	\\[0.3em]
       ...		& ...			& ...	&	...		\\[0.3em]
       \sum x_{ck} x_{1k} 	& \sum x_{ck} x_{2k} 	& ...	& \sum x_{ck}^{2}	\\[0.3em]
\end{pmatrix}
\]
\[
=(1/N)
\begin{pmatrix}
       \sum X_{1k}^{2} 	& \sum X_{1k} X_{2k} 	& ...	& \sum X_{1k} X_{ck}	\\[0.3em]
       \sum X_{2k} X_{1k} 	& \sum X_{2k} X_{2k} 	& ...	& \sum X_{2k} X_{ck}	\\[0.3em]
       ...		& ...			& ...	&	...		\\[0.3em]
       \sum X_{ck} X_{1k} 	& \sum X_{ck} X_{2k} 	& ...	& \sum X_{ck}^{2}	\\[0.3em]
\end{pmatrix}
\]
\[
=(1/N)
\begin{pmatrix}
       \langle X_1, X_1 \rangle	& \langle X_1, X_2 \rangle 	& ...	& \langle X_1, X_c	\rangle \\[0.3em]
       \langle X_2, X_1 \rangle 	& \langle X_2, X_2 \rangle	& ...	& \langle X_2, X_c	\rangle \\[0.3em]
       ...		& ...			& ...	&	...		\\[0.3em]
       \langle X_c, X_1 \rangle	& \langle X_c, X_2 \rangle 	& ...	& \langle X_c, X_c\rangle	\\[0.3em]
\end{pmatrix}
\]
Let \textbf{X}=$\begin{pmatrix} R & G & B \end{pmatrix}$, R,G,B are column vectors which have been normalized (detrend, zero mean).
\[
R=
\begin{pmatrix}
       R_0\\[0.3em]
       R_1\\[0.3em]
       R_2\\[0.3em]
       R_3\\[0.3em]
		. \\[0.3em]
       R_{n-1}\\[0.3em]
\end{pmatrix}
, G=
\begin{pmatrix}
       G_0\\[0.3em]
       G_1\\[0.3em]
       G_2\\[0.3em]
       G_3\\[0.3em]
		. \\[0.3em]
       G_{n-1}\\[0.3em]
\end{pmatrix}
, B=
\begin{pmatrix}
       B_0\\[0.3em]
       B_1\\[0.3em]
       B_2\\[0.3em]
       B_3\\[0.3em]
		. \\[0.3em]
       B_{n-1}\\[0.3em]
\end{pmatrix}
\]

\[
\Sigma(\textbf{X})=(1/N)
\begin{pmatrix}
       \langle R, R \rangle	& \langle R, G \rangle 	& \langle R, B	\rangle \\[0.3em]
       \langle G, R \rangle 	& \langle G, G \rangle		& \langle G, B	\rangle \\[0.3em]
       \langle B, R \rangle	& \langle B, G \rangle 	& \langle B, B	\rangle	\\[0.3em]
\end{pmatrix}
=
\begin{pmatrix}
       \sigma_R^2              & \langle R, G \rangle 	&  \langle R, B	\rangle \\[0.3em]
       \langle G, R \rangle 	& \sigma_G^2	            & \langle G, B	\rangle \\[0.3em]
       \langle B, R \rangle	& \langle B, G \rangle 	& \sigma_B^2	\\[0.3em]
\end{pmatrix}
\]
\[
=
\begin{pmatrix}
       R_0 & R_1 & R_2 & R_3 ... & R_{n-1} \\[0.3em]
       G_0 & G_1 & G_2 & G_3 ... & G_{n-1} \\[0.3em]
       B_0 & B_1 & B_2 & B_3 ... & B_{n-1} \\[0.3em]
\end{pmatrix}
\begin{pmatrix}
       R_0     & G_0     & B_0\\[0.3em]
       R_1     & G_1     & B_1\\[0.3em]
       R_2     & G_2     & B_2\\[0.3em]
       R_3     & G_3     & B_3\\[0.3em]
		.      & .       & .  \\[0.3em]
       R_{n-1} & G_{n-1} & B_{n-1}\\[0.3em]
\end{pmatrix}
=
\begin{pmatrix}
       R \\[0.3em]
       G \\[0.3em]
       B \\[0.3em]
\end{pmatrix}
\begin{pmatrix}
       R & G & B\\[0.3em]
\end{pmatrix}
\]
The time complexity is $O(N^2),  (C_2^N) = O(N^2/2)$.
\end{compactitem}

%----------------------------------------------------------------------------------------

\section{In Closing}

You have reached the end of this mini-guide. You can now rename or overwrite this pdf file and begin writing your own `\texttt{Chapter1.tex}' and the rest of your thesis. The easy work of setting up the structure and framework has been taken care of for you. It's now your job to fill it out!

Good luck and have lots of fun!

\begin{flushright}
Guide written by ---\\
Sunil Patel: \href{http://www.sunilpatel.co.uk}{www.sunilpatel.co.uk}
\end{flushright}
